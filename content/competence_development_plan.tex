%----------------------------------------------------------------------------
\chapter{\DevelopementPlan}
%----------------------------------------------------------------------------
\section{Fejlesztés célja és kiindulópontja}
%----------------------------------------------------------------------------

A kompetenciaelemzés során világossá vált számomra, hogy vezetőként már most is számos erős 
területtel rendelkezem, ugyanakkor a fejlődés lehetősége folyamatos.
Célom, hogy tudatosan továbbfejlesszem azokat a készségeimet, amelyek a mindennapi vezetői 
munkám során még hatékonyabbá, kiegyensúlyozottabbá és inspirálóbbá tehetnek.
A fejlesztési terv alapját az önértékelő teszt eredményei adják, valamint a saját 
tapasztalataim és visszajelzéseim a kollégáimtól.

Kiemelten szeretnék foglalkozni három fő fejlesztési területtel:

\begin{enumerate}
    \item A stresszkezelés és energiamenedzsment tudatosabb egyensúlyba hozása.
    \item A jutalmazás és elismerés strukturáltabb, tudatosabb alkalmazása a csapatomban.
    \item Az önfejlesztés és inspiráció hosszú távú fenntartása, hogy vezetőként ne csak irányítsak, hanem példát is mutassak.
\end{enumerate}

%----------------------------------------------------------------------------
\section{Fejlesztési területek és célkitűzések}
%----------------------------------------------------------------------------

\subsection{Stresszkezelés és energiamenedzsment}

\textbf{Kiinduló helyzet:} \\
A munkám során gyakran kezelek több, párhuzamosan futó projektet, amelyekhez magas szintű 
koncentráció és folyamatos felelősségvállalás társul.
Noha általában jól reagálok stresszes helyzetekre, időnként hajlamos vagyok túlvállalni magam, 
ami hosszú távon mentális fáradtsághoz vezethet.

\textbf{Fejlesztési cél:} \\
Tudatos időbeosztás és energiamenedzsment kialakítása annak érdekében, hogy a hatékonyság mellett 
a regenerálódásra is elegendő idő jusson.

\textbf{Fejlesztési lépések:}
\begin{itemize}
    \item A heti munkabeosztás átnézése és a „pufferidők” beépítése a határidők közé.
    \item Rendszeres pihenőidők beiktatása (pl. heti 1 nap teljes offline nap).
    \item Relaxációs és koncentrációs technikák (pl. légzőgyakorlatok, rövid meditáció) tudatos használata.
    \item A sport és fizikai aktivitás beépítése a heti rutinba – nemcsak fizikai, hanem mentális feltöltődés céljából.
\end{itemize}

\textbf{Eredményességi mutató:} \\
Érezhető energiaszint-növekedés, kevesebb túlhajszoltság, kiegyensúlyozottabb döntéshozatal a projektek során.

\subsection{Jutalmazás és elismerés a csapatban}

\textbf{Kiinduló helyzet:} \\
A vezetői munkám során elsősorban személyes példamutatással és szakmai támogatással motiválom a kollégáimat.
Bár ez hatékony, úgy érzem, hogy a strukturáltabb visszajelzés és elismerésrendszer erősíthetné 
a csapatkohéziót és az egyéni teljesítményt.

\textbf{Fejlesztési cél:} \\
Olyan, tudatosan felépített elismerési rendszer kialakítása, amely nemcsak szakmai, hanem emberi 
szinten is megerősíti a csapattagokat.

\textbf{Fejlesztési lépések:}
\begin{itemize}
    \item Havi rendszerességű személyes visszajelző beszélgetések bevezetése.
    \item Egyéni és csoportszintű teljesítmények elismerése szóban és írásban.
    \item Kis léptékű, de rendszeres jutalmazási formák (pl. közös program, kiemelés meetingeken).
    \item A csapattagok bevonása a motivációs eszközök közös megtervezésébe.
\end{itemize}

\textbf{Eredményességi mutató:} \\
Növekvő csapatmotiváció, pozitívabb visszajelzések a munkahelyi légkörről, alacsonyabb fluktuáció és jobb együttműködés.

\subsection{Önismeret és inspiráció fenntartása}

\textbf{Kiinduló helyzet:} \\
Vezetőként fontos számomra, hogy folyamatosan fejlődjek, nyitott maradjak az új megoldásokra, és ne ragadjak bele a rutinba.
Tudom, hogy a hosszú távú önfejlesztés kulcsa az, ha én magam is folyamatosan tanulok, 
fejlődök és új impulzusokat keresek.

\textbf{Fejlesztési cél:} \\
Az önismeret és a szakmai fejlődés tudatos fenntartása, valamint az motiváció átadása a csapatom számára.

\textbf{Fejlesztési lépések:}
\begin{itemize}
    \item Rendszeres önreflexiós naplóvezetés (havonta egyszer).
    \item Évente legalább két szakmai konferencia vagy tréning látogatása.
    \item Inspiráló szakirodalmak, vezetéselméleti könyvek rendszeres olvasása.
    \item Mentori kapcsolat kialakítása vagy szakmai tapasztalatcsere egy hasonló területen dolgozó vezetővel.
\end{itemize}

\textbf{Eredményességi mutató:} \\
Tartós szakmai fejlődés, fokozódó motiváció, valamint a csapat inspirálása és fejlődése a saját példámon keresztül.
%----------------------------------------------------------------------------
\begin{figure}[H]
	\centering
    \includegraphics[width=30mm, keepaspectratio]{figures/self_knowledge.jpg}
    \caption{Önismeret kreatív ábrázolása}
    \label {fig:self_knowledge}
\end{figure}
%----------------------------------------------------------------------------
\section{Összegzés}
%----------------------------------------------------------------------------

A fejlesztési tervem célja, hogy a vezetői kompetenciáimat a következő években tudatosan továbbfejlesszem, 
és stabil alapokra helyezzem a hosszú távú szakmai hatékonyságomat.
Úgy érzem, a legfontosabb lépés az, hogy megtartsam azt az önismeretet és felelősségtudatot, 
amely eddig is jellemezte a munkámat, miközben új szintre emelem a csapatom motiválását és a saját energiamenedzsmentemet.

Meggyőződésem, hogy a vezetés nem pozíció, hanem szemlélet és a fejlődéshez való nyitottság az, 
ami igazán jó vezetővé teheti az embert.
