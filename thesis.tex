% !TeX root = ./thesis.tex
% !TeX spellcheck = hu_HU
% !TeX encoding = UTF-8
% !TeX program = pdflatex
% !TeX TXS-program:compile = txs:///python//usr/bin/python3
\documentclass[12pt,a4paper,oneside]{report}             % Egyoldalas (javasolt)
%\documentclass[11pt,a4paper,twoside,openright]{report}  % Duplex
\input{include/packages}
%--------------------------------------------------------------------------------------
\usepackage{float}
\usepackage{hyperref}
\usepackage{url}
\usepackage{caption}
\usepackage{graphicx} % Képek kezeléséhez
\usepackage{listings} % Kódrészletekhez
\usepackage{booktabs} % Táblázatokhoz
\usepackage{comment}
\usepackage[magyar]{babel}
\usepackage{amsmath}
\usepackage{ragged2e}
\usepackage{array}
\usepackage{times}
\usepackage{fancyhdr}
\usepackage[a4paper, left=3cm, right=2.54cm, top=2.54cm, bottom=2.54cm]{geometry}
% Fejléc és lábléc
\pagestyle{fancy}
\fancyhf{}
\fancyhead[L]{\textnormal{\thetitle}}
\fancyhead[C]{\szerzoNeptun}
\fancyfoot[C]{\thepage}
\renewcommand{\headrulewidth}{0.5pt}
\renewcommand{\footrulewidth}{0pt}

\usepackage{titlesec}
% 1. szint: Decimális számozás, 3.1.3-ban meghatározott stílus
\titleformat{\section}[block]
  {\bfseries\fontsize{16pt}{19.2pt}\selectfont} % Félkövér, 16pt betűméret
  {\thesection.} % Számozás
  {1em} % Címszöveg előtti távolság
  {} % Címszöveg formázása

\titlespacing{\section}
  {0pt} % Bal oldali behúzás
  {24pt} % Cím előtti térköz
  {18pt} % Cím utáni térköz

% 2. szint: Decimális számozás, stílus szerint
\titleformat{\subsection}[block]
  {\bfseries\fontsize{14pt}{16.8pt}\selectfont} % Félkövér, 14pt betűméret
  {\thesubsection.}
  {1em}
  {}

\titlespacing{\subsection}
  {0pt}
  {18pt}
  {12pt}

% 3. szint: Decimális számozás, stílus szerint
\titleformat{\subsubsection}[block]
  {\normalfont\fontsize{14pt}{16.8pt}\selectfont} % Normál, 14pt betűméret
  {\thesubsubsection.}
  {1em}
  {}

\titlespacing{\subsubsection}
  {0pt}
  {12pt}
  {6pt}

% 4. és további szintek: egyéni meghatározás
\titleformat{\subsubsubsection}[block]
  {\normalfont\fontsize{14pt}{16.8pt}\selectfont} % Normál, 14pt betűméret
  {\thesubsubsubsection.}
  {1em}
  {}

\titlespacing{\subsubsubsection}
  {0pt}
  {12pt}
  {6pt}

% Alapértelmezett képaláírás beállítás
\captionsetup{
  font=bf, % Félkövér
  labelfont=bf,
  format=plain,
  justification=centering, % Középre igazítás
  textfont={bf,rm}, % Times New Roman betűtípus
  labelsep=period, % Pont a típus és a cím között
  %skip=6pt, % Cím előtti térköz
  %belowskip=18pt % Cím utáni térköz
}

% Képeknél és ábráknál a ”képaláírás” az objektum alá, táblázatoknál és kódrészleteknél az objektum fölé kerüljön
%\captionsetup[figure]{position=bottom}
%\captionsetup[table]{position=top}
%\captionsetup[lstlisting]{position=top}


%--------------------------------------------------------------------------------------
\newcommand{\szerzoVezeteknev}{Székely}
\newcommand{\szerzoKeresztnev}{Dániel}
\newcommand{\szerzoNeptun}{JAXC3C}

\newcommand{\szakirany}{} % Informatikusoknál nincs szakirány. Villamosmérnököknél: Automatizálás (\aut) vagy Infokommunikáció (\infokom).

\newcommand{\konzulensAMegszolitas}{}
\newcommand{\konzulensAVezeteknev}{Paál}
\newcommand{\konzulensAKeresztnev}{Dávid}
\newcommand{\konzulensBMegszolitas}{}
\newcommand{\konzulensBVezeteknev}{Tamás}
\newcommand{\konzulensBKeresztnev}{Dávid}
\newcommand{\konzulensCMegszolitas}{}
\newcommand{\konzulensCVezeteknev}{}
\newcommand{\konzulensCKeresztnev}{}

\newcommand{\cim}{Mitől leszek jó vezető? Egyéni kompetencia-elemzés és fejlesztési terv} % Cím
\newcommand{\tanszek}{\szeit} % informatika (\szeit), automatizálási (\szeaut) vagy távközlési (\szetat)
\newcommand{\szak}{\infoMSc} % Mérnökinformatikus BSc (\infoMsc), MSc (\infoMsc), Gazdaságinformatikus BSc (\gazdInfoBsc), MSc (\gazdInfoMsc), vagy Villamosmérnöki BSc (\villBSc), MSc (\villMSc)

\include{include/variables}
% Beállítások magyar nyelvű dolgozathoz
\input{include/thesis-hu}
% Settings for English documents
%\input{include/thesis-en}


\newcommand{\szerzoMeta}{\szerzoVezeteknev{} \szerzoKeresztnev} % egy szerző esetén TODO@FMA két szerző
\newcommand{\doktipus}{\szakdolgozat} % Dokumentum típusa (\szakdolgozat, \diplomaterv vagy \dolgozat)

\input{include/preamble} % beállítások, nem kell vele foglalkoznod remélhetőleg, de ha valami latex hekkelésre vagy új parancsra van szükséged annak itt a helye

% A default szöveg legyen times new roman 12-es


%--------------------------------------------------------------------------------------
% Itt kezdődik a dolgozat
%--------------------------------------------------------------------------------------
% Másfeles sorköz
\newcommand{\masfelessorkoz}{\renewcommand{\baselinestretch}{1.24}\small\normalsize}
%--------------------------------------------------------------------------------------

\begin{document}

\selectthesislanguage

% Külső borító, minta kötéshez - csak elektronikus leadás esetén eltávolítandó
%~~~~~~~~~~~~~~~~~~~~~~~~~~~~~~~~~~~~~~~~~~~~~~~~~~~~~~~~~~~~~~~~~~~~~~~~~~~~~~~~~~~~~~
%\include{include/outercover}

% Címoldal 
%~~~~~~~~~~~~~~~~~~~~~~~~~~~~~~~~~~~~~~~~~~~~~~~~~~~~~~~~~~~~~~~~~~~~~~~~~~~~~~~~~~~~~~
\include{include/titlepage}

%TODO Feladatkiíró lap helye, csak a nyomtatott verzióba kerül az eredeti példány
%~~~~~~~~~~~~~~~~~~~~~~~~~~~~~~~~~~~~~~~~~~~~~~~~~~~~~~~~~~~~~~~~~~~~~~~~~~~~~~~~~~~~~~
\pagenumbering{gobble}
%--------------------------------------------------------------------------------------
% Feladatkiiras (a tanszeken atveheto, kinyomtatott valtozat)
%--------------------------------------------------------------------------------------
\clearpage

% PDF formátumú leírás esetén
%\includepdf{figures/start.pdf}

% Képfájlokhoz
%\includegraphics*[width=\linewidth]{figures/start.png}

% Nyilatkozat és Kivonat
%~~~~~~~~~~~~~~~~~~~~~~~~~~~~~~~~~~~~~~~~~~~~~~~~~~~~~~~~~~~~~~~~~~~~~~~~~~~~~~~~~~~~~~
\masfelessorkoz{
% Tartalomjegyzék
%~~~~~~~~~~~~~~~~~~~~~~~~~~~~~~~~~~~~~~~~~~~~~~~~~~~~~~~~~~~~~~~~~~~~~~~~~~~~~~~~~~~~~~
\tableofcontents\vfill

% A dolgozat lényegi része
%~~~~~~~~~~~~~~~~~~~~~~~~~~~~~~~~~~~~~~~~~~~~~~~~~~~~~~~~~~~~~~~~~~~~~~~~~~~~~~~~~~~~~~
\newpage
\pagenumbering{arabic}
\setcounter{page}{1}

% Saját munka

%----------------------------------------------------------------------------
\chapter{\bevezetes}
%----------------------------------------------------------------------------

%----------------------------------------------------------------------------
\section{Bevezetés}
%----------------------------------------------------------------------------

%----------------------------------------------------------------------------
\chapter{\Literature}
%----------------------------------------------------------------------------
\section{A vezetés és menedzsment fogalmi elhatárolása}
%----------------------------------------------------------------------------

A „vezetés” és a „menedzsment” kifejezéseket a mindennapokban gyakran szinonimaként használjuk, 
mégis eltérő tartalmakat hordoznak. 
A menedzsment inkább a szervezéshez, tervezéshez és az erőforrások hatékony felhasználásához kapcsolódik, 
míg a vezetés ennél mélyebbre nyúl: az emberek irányításáról, motiválásáról és inspirálásáról szól.
Ahogy Peter Drucker is megfogalmazta: 
„A menedzser a dolgokat jól csinálja, a vezető pedig a jó dolgokat csinálja.” \cite{drucker1954}
A vezető tehát nem pusztán végrehajt, hanem irányt mutat, értéket közvetít és jövőképet ad a csapatnak. 

A modern vezetéselméletek szerint a legjobb vezetők képesek egyensúlyt teremteni a strukturált 
menedzseri gondolkodás és az emberközpontú vezetői szemlélet között \cite{mintzberg1975}. 
A hatékony vezetés így nem csupán a célok elérését jelenti, hanem az emberek fejlődésének támogatását is, 
amely a szervezeti siker egyik legfontosabb tényezője.

%----------------------------------------------------------------------------
\section{A kompetencia fogalma és típusai}
%----------------------------------------------------------------------------

A kompetencia olyan összetett jellemző, amely magában foglalja az egyén tudását, 
készségeit, attitűdjeit és motivációját - vagyis mindazt, ami lehetővé teszi, 
hogy valaki eredményesen lássa el a feladatait \cite{spencer1993}. 
A kompetencia nem veleszületett adottság, hanem folyamatosan fejleszthető és tanulható tulajdonság.

Három fő típusa különíthető el:

\begin{itemize}
    \item \textbf{Szakmai kompetenciák:} a munkakörhöz kapcsolódó technikai ismeretek és gyakorlati készségek.
    \item \textbf{Személyes kompetenciák:} önismeret, döntéshozatal, felelősségvállalás, problémamegoldás.
    \item \textbf{Szociális (személyközi) kompetenciák:} kommunikáció, együttműködés, empátia, vezetői befolyás.
\end{itemize}

A vezetői kompetenciák ezek ötvözetéből épülnek fel, és azt mutatják meg, hogy egy vezető milyen mértékben képes hatékonyan irányítani embereket, motiválni őket és a szervezeti célokat a csapat céljaival összehangolni \cite{boyatzis1982}.
%----------------------------------------------------------------------------
\begin{figure}[H]
	\centering
    \includegraphics[width=100mm, keepaspectratio]{figures/competence.jpg}
    \caption{Kompetencia típusok vizualizációja} 
    \label {fig:competence}
\end{figure}
%----------------------------------------------------------------------------
\section{A vezetői kompetenciák főbb modelljei}
%----------------------------------------------------------------------------

A vezetői kompetenciák kutatása több évtizedes múltra tekint vissza. 
Robert L. Katz (1955) három alapvető vezetői készséget különített el \cite{katz1955}:

\begin{itemize}
    \item \textbf{Technikai készségek} - a szakmai ismeretek és eszközhasználat képessége.
    \item \textbf{Emberi készségek} - a kommunikáció és együttműködés képessége.
    \item \textbf{Koncepcionális készségek} - a stratégiai gondolkodás, rendszerszintű látásmód.
\end{itemize}

A későbbi modellek, mint például Daniel Goleman érzelmi intelligencia alapú vezetéselmélete, 
kiegészítették ezt azzal, hogy a sikeres vezetés nem csak kognitív képességeken, 
hanem érzelmi tudatosságon, önszabályozáson, empátián és motiváción is múlik \cite{goleman1998}. 

Az európai vezetői kompetenciakeretek (pl. \textit{European Competency Framework for Managers}) 
hangsúlyozzák a felelős döntéshozatalt, az etikus viselkedést, valamint a csapatfejlesztést és innovációt is mint kulcskompetenciákat \cite{ecfm2010}.

Összességében elmondható, hogy a vezetői kompetenciák nemcsak a „mit tudok” szintjén, 
hanem a „hogyan működöm” szinten is meghatározzák a vezető hatékonyságát. 
Egy jó vezető nemcsak irányít, hanem képes inspirálni, példát mutatni és fejlődni önmaga és a csapata érdekében.
%----------------------------------------------------------------------------
\begin{figure}[H]
	\centering
    \includegraphics[width=100mm, keepaspectratio]{figures/inspiration.jpg}
    \caption{Inspiráció kreatív ábrázolása}
    \label {fig:inspiration}
\end{figure}
%----------------------------------------------------------------------------
%---------------------------------------------------------------------------
\chapter{\Introspection}
%----------------------------------------------------------------------------
\section{A teszt bemutatása és célja}
%----------------------------------------------------------------------------

A 84 kérdésből álló PAMS önértékelő kérdőív kitöltésével saját vezetői 
és személyes kompetenciáimat szerettem volna feltérképezni.
A teszt három fő területre fókuszál: egyéni, személyközi és csoportos készségekre.
Célom az volt, hogy reális képet kapjak arról, milyen erősségekkel rendelkezem vezetőként, 
és mely területeken érdemes még fejlődnöm.

A kérdőív 1-től 6-ig terjedő skálán értékeli az egyes kompetenciákat, ahol a 6-os a legmagasabb szintet jelöli.
A válaszaim túlnyomó része 6-os értéket kapott, ami azt jelzi, hogy tudatosan törekszem a fejlődésre, 
és már most is erős önismerettel, következetességgel és felelősségvállalással dolgozom.

%----------------------------------------------------------------------------
\section{Általános értékelés}
%----------------------------------------------------------------------------

Összességében az eredmények alapján magas szintű vezetői és személyes kompetenciákkal rendelkezem.
Különösen erősnek érzem magam a kommunikáció, a problémamegoldás, a csapatvezetés és az innovatív gondolkodás területein.
A legtöbb válasz esetében 6-os értéket adtam, ami azt tükrözi, hogy stabilan, tudatosan és értékalapúan működöm.
A céljaimat rendszerint végigviszem, képes vagyok másokat motiválni, és jól reagálok stresszes vagy változó helyzetekben is.

%----------------------------------------------------------------------------
\section{Részletes kompetenciaelemzés}
%----------------------------------------------------------------------------

\subsection{Önismeret és személyes hatékonyság}

Fontosnak tartom az önreflexiót és az önismeretet, hiszen ezek nélkül nem lehet hatékonyan vezetni.
Tisztában vagyok az erősségeimmel és a korlátaimmal is, és nem félek a visszajelzésektől.
Igyekszem folyamatosan fejleszteni magam, tanulni a tapasztalataimból, 
és javítani azokon a területeken, ahol érzem, hogy van még hova fejlődni.
Ezen a téren nagyon magas pontszámot adtam magamnak (5,9/6), mert valóban tudatosan figyelek önmagam működésére.

\subsection{Stresszkezelés és időgazdálkodás}

Úgy érzem, jól kezelem a stresszt, és tudatosan figyelek arra, hogy ne csak hatékonyan dolgozzak, hanem regenerálódjak is.
A feladataimat igyekszem mindig priorizálni, és nem hagyom, hogy a felesleges terhelés elvigye az energiámat.
Viszont elismerem, hogy a pihenésre és kikapcsolódásra néha több 
figyelmet kellene fordítanom, mert hajlamos vagyok túlvállalni magam, ha fontos projektről van szó.

\subsection{Problémamegoldás és döntéshozatal}

Erősségemnek tartom a rendszerszintű és logikus gondolkodást.
Döntéseimet nem hirtelen hozom meg, hanem mindig több lehetőséget mérlegelek, és igyekszem a legjobb megoldást választani.
Ugyanakkor nem félek kockázatot vállalni, ha azt érzem, hogy az a fejlődés vagy a csapat érdekeit szolgálja.
A kreatív gondolkodás nálam természetes része a problémamegoldásnak, de tudatosan figyelek arra is, 
hogy a megoldások ne csak innovatívak, hanem kivitelezhetők is legyenek.

\subsection{Kreativitás és innováció}

A munkám során sokszor kerülök olyan helyzetbe, ahol gyorsan, kreatívan kell reagálni.
Szeretek új megoldásokat kipróbálni, akár technikai, akár szervezési szinten.
Fontosnak tartom, hogy a csapatom is merjen kísérletezni és fejlődni, mert hiszek abban, 
hogy az innováció csak így születhet meg.
Ezen a területen is magas értékelést adtam magamnak (5,9/6), mivel a kreatív gondolkodás 
és a megújulás képessége alapvető része a vezetői szemléletemnek.

\subsection{Kommunikáció és visszajelzés}

A kommunikáció mindig is központi szerepet játszott a munkámban.
Törekszem arra, hogy világosan, érthetően és empatikusan kommunikáljak, akár egy csapattaggal, akár ügyféllel beszélek.
A visszajelzéseket nem csak adom, hanem kérem is, mert ezekből tudok tanulni.
Fontos számomra, hogy a problémákat mindig nyíltan, őszintén, 
de tisztelettel kezeljem - ez szerintem az egyik legfontosabb vezetői kompetencia, amit sikerült tudatosan kialakítanom.

\subsection{Motiváció és teljesítménymenedzsment}

A csapat motiválása nálam elsősorban emberi kapcsolatokon és példamutatáson alapul.
Sokszor személyesen is próbálok inspirálni másokat, nemcsak utasításokkal, hanem bizalommal és támogatással.
Bevallom, a formális jutalmazási rendszert kevésbé alkalmazom, mert inkább 
a szakmai elismerésben és a fejlődés lehetőségében hiszek.
Ugyanakkor érzem, hogy ezen a területen érdemes lenne tudatosabban használni 
a jutalmazás és elismerés eszközeit, mert ez még hatékonyabbá teheti a csapatműködést.

\subsection{Csapatvezetés és együttműködés}

Szeretek csapatban dolgozni, és hiszek abban, hogy a legjobb eredmények mindig közösen születnek.
Tudatosan figyelek arra, hogy mindenki értse a célokat, és lássa, miért fontos a saját szerepe a közös sikerben.
Nyitott vagyok a véleményekre, és igyekszem olyan légkört teremteni, ahol mindenki bátran megoszthatja a gondolatait.
A csapatvezetés terén stabilnak érzem magam, mert nemcsak irányítok, hanem fejlesztek és támogatok is.

\subsection{Változáskezelés és inspiráció}

Hiszek abban, hogy a vezetés nem a hatalomról, hanem a példamutatásról és a hatásról szól.
A változásokat nem kényszerként, hanem lehetőségként élem meg, és ezt próbálom másokban is tudatosítani.
Sokszor ösztönösen igyekszem inspirálni a környezetemet, 
mert szerintem csak így lehet hosszú távon fejlődni - ha az emberek hisznek abban, amit csinálnak.
Ez a kompetenciaterület az egyik legerősebb oldalam, szinte minden válaszomnál 6-os értéket adtam.

%----------------------------------------------------------------------------
\section{Összegzés és önreflexió}
%----------------------------------------------------------------------------

A PAMS teszt megerősítette, amit eddig is éreztem: erős önismerettel, empatikus és tudatos vezetői szemlélettel rendelkezem.
A legtöbb kompetenciaterületen magas szintet érek el, különösen a kommunikáció, 
csapatvezetés, innováció és problémamegoldás terén.
Fejlesztendő területként leginkább a jutalmazási rendszer tudatosabb használatát és a stresszkezelés kiegyensúlyozását látom.

Úgy érzem, jó úton haladok abban, hogy a vezetői szerepemet még tudatosabban, 
kiegyensúlyozottabban és inspirálóbban tudjam betölteni.
A következő fejezetben ennek alapján készítem el a fejlesztési tervemet, amely 
konkrét célokat és lépéseket tartalmaz a személyes és vezetői fejlődésem támogatására.

%----------------------------------------------------------------------------
\chapter{\DevelopementPlan}
%----------------------------------------------------------------------------
\section{Fejlesztés célja és kiindulópontja}
%----------------------------------------------------------------------------

A kompetenciaelemzés során világossá vált számomra, hogy vezetőként már most is számos erős 
területtel rendelkezem, ugyanakkor a fejlődés lehetősége folyamatos.
Célom, hogy tudatosan továbbfejlesszem azokat a készségeimet, amelyek a mindennapi vezetői 
munkám során még hatékonyabbá, kiegyensúlyozottabbá és inspirálóbbá tehetnek.
A fejlesztési terv alapját az önértékelő teszt eredményei adják, valamint a saját 
tapasztalataim és visszajelzéseim a kollégáimtól.

Kiemelten szeretnék foglalkozni három fő fejlesztési területtel:

\begin{enumerate}
    \item A stresszkezelés és energiamenedzsment tudatosabb egyensúlyba hozása.
    \item A jutalmazás és elismerés strukturáltabb, tudatosabb alkalmazása a csapatomban.
    \item Az önfejlesztés és inspiráció hosszú távú fenntartása, hogy vezetőként ne csak irányítsak, hanem példát is mutassak.
\end{enumerate}

%----------------------------------------------------------------------------
\section{Fejlesztési területek és célkitűzések}
%----------------------------------------------------------------------------

\subsection{Stresszkezelés és energiamenedzsment}

\textbf{Kiinduló helyzet:} \\
A munkám során gyakran kezelek több, párhuzamosan futó projektet, amelyekhez magas szintű 
koncentráció és folyamatos felelősségvállalás társul.
Noha általában jól reagálok stresszes helyzetekre, időnként hajlamos vagyok túlvállalni magam, 
ami hosszú távon mentális fáradtsághoz vezethet.

\textbf{Fejlesztési cél:} \\
Tudatos időbeosztás és energiamenedzsment kialakítása annak érdekében, hogy a hatékonyság mellett 
a regenerálódásra is elegendő idő jusson.

\textbf{Fejlesztési lépések:}
\begin{itemize}
    \item A heti munkabeosztás átnézése és a „pufferidők” beépítése a határidők közé.
    \item Rendszeres pihenőidők beiktatása (pl. heti 1 nap teljes offline nap).
    \item Relaxációs és koncentrációs technikák (pl. légzőgyakorlatok, rövid meditáció) tudatos használata.
    \item A sport és fizikai aktivitás beépítése a heti rutinba – nemcsak fizikai, hanem mentális feltöltődés céljából.
\end{itemize}

\textbf{Eredményességi mutató:} \\
Érezhető energiaszint-növekedés, kevesebb túlhajszoltság, kiegyensúlyozottabb döntéshozatal a projektek során.

\subsection{Jutalmazás és elismerés a csapatban}

\textbf{Kiinduló helyzet:} \\
A vezetői munkám során elsősorban személyes példamutatással és szakmai támogatással motiválom a kollégáimat.
Bár ez hatékony, úgy érzem, hogy a strukturáltabb visszajelzés és elismerésrendszer erősíthetné 
a csapatkohéziót és az egyéni teljesítményt.

\textbf{Fejlesztési cél:} \\
Olyan, tudatosan felépített elismerési rendszer kialakítása, amely nemcsak szakmai, hanem emberi 
szinten is megerősíti a csapattagokat.

\textbf{Fejlesztési lépések:}
\begin{itemize}
    \item Havi rendszerességű személyes visszajelző beszélgetések bevezetése.
    \item Egyéni és csoportszintű teljesítmények elismerése szóban és írásban.
    \item Kis léptékű, de rendszeres jutalmazási formák (pl. közös program, kiemelés meetingeken).
    \item A csapattagok bevonása a motivációs eszközök közös megtervezésébe.
\end{itemize}

\textbf{Eredményességi mutató:} \\
Növekvő csapatmotiváció, pozitívabb visszajelzések a munkahelyi légkörről, alacsonyabb fluktuáció és jobb együttműködés.

\subsection{Önismeret és inspiráció fenntartása}

\textbf{Kiinduló helyzet:} \\
Vezetőként fontos számomra, hogy folyamatosan fejlődjek, nyitott maradjak az új megoldásokra, és ne ragadjak bele a rutinba.
Tudom, hogy a hosszú távú önfejlesztés kulcsa az, ha én magam is folyamatosan tanulok, 
fejlődök és új impulzusokat keresek.

\textbf{Fejlesztési cél:} \\
Az önismeret és a szakmai fejlődés tudatos fenntartása, valamint az motiváció átadása a csapatom számára.

\textbf{Fejlesztési lépések:}
\begin{itemize}
    \item Rendszeres önreflexiós naplóvezetés (havonta egyszer).
    \item Évente legalább két szakmai konferencia vagy tréning látogatása.
    \item Inspiráló szakirodalmak, vezetéselméleti könyvek rendszeres olvasása.
    \item Mentori kapcsolat kialakítása vagy szakmai tapasztalatcsere egy hasonló területen dolgozó vezetővel.
\end{itemize}

\textbf{Eredményességi mutató:} \\
Tartós szakmai fejlődés, fokozódó motiváció, valamint a csapat inspirálása és fejlődése a saját példámon keresztül.
%----------------------------------------------------------------------------
\begin{figure}[H]
	\centering
    \includegraphics[width=30mm, keepaspectratio]{figures/self_knowledge.jpg}
    \caption{Önismeret kreatív ábrázolása}
    \label {fig:self_knowledge}
\end{figure}
%----------------------------------------------------------------------------
\section{Összegzés}
%----------------------------------------------------------------------------

A fejlesztési tervem célja, hogy a vezetői kompetenciáimat a következő években tudatosan továbbfejlesszem, 
és stabil alapokra helyezzem a hosszú távú szakmai hatékonyságomat.
Úgy érzem, a legfontosabb lépés az, hogy megtartsam azt az önismeretet és felelősségtudatot, 
amely eddig is jellemezte a munkámat, miközben új szintre emelem a csapatom motiválását és a saját energiamenedzsmentemet.

Meggyőződésem, hogy a vezetés nem pozíció, hanem szemlélet és a fejlődéshez való nyitottság az, 
ami igazán jó vezetővé teheti az embert.


% Ábrák listája - a word-ös sablon szerint szükséges
%~~~~~~~~~~~~~~~~~~~~~~~~~~~~~~~~~~~~~~~~~~~~~~~~~~~~~~~~~~~~~~~~~~~~~~~~~~~~~~~~~~~~~~
\listoffigures\addcontentsline{toc}{chapter}{\listfigurename}

% Táblázatok listája - opcionális
%~~~~~~~~~~~~~~~~~~~~~~~~~~~~~~~~~~~~~~~~~~~~~~~~~~~~~~~~~~~~~~~~~~~~~~~~~~~~~~~~~~~~~~
\listoftables\addcontentsline{toc}{chapter}{\listtablename}

% Irodalomjegyzék
%~~~~~~~~~~~~~~~~~~~~~~~~~~~~~~~~~~~~~~~~~~~~~~~~~~~~~~~~~~~~~~~~~~~~~~~~~~~~~~~~~~~~~~
\addcontentsline{toc}{chapter}{\bibname}
\bibliography{bib/references}

% Függelékek
%~~~~~~~~~~~~~~~~~~~~~~~~~~~~~~~~~~~~~~~~~~~~~~~~~~~~~~~~~~~~~~~~~~~~~~~~~~~~~~~~~~~~~~
%\include{content/appendices}

%\label{page:last}
}
\end{document}